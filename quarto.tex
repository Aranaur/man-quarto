% Options for packages loaded elsewhere
\PassOptionsToPackage{unicode}{hyperref}
\PassOptionsToPackage{hyphens}{url}
\PassOptionsToPackage{dvipsnames,svgnames,x11names}{xcolor}
%
\documentclass[
  letterpaper,
  DIV=11,
  numbers=noendperiod,
  oneside]{scrartcl}

\usepackage{amsmath,amssymb}
\usepackage{lmodern}
\usepackage{iftex}
\ifPDFTeX
  \usepackage[T1]{fontenc}
  \usepackage[utf8]{inputenc}
  \usepackage{textcomp} % provide euro and other symbols
\else % if luatex or xetex
  \usepackage{unicode-math}
  \defaultfontfeatures{Scale=MatchLowercase}
  \defaultfontfeatures[\rmfamily]{Ligatures=TeX,Scale=1}
\fi
% Use upquote if available, for straight quotes in verbatim environments
\IfFileExists{upquote.sty}{\usepackage{upquote}}{}
\IfFileExists{microtype.sty}{% use microtype if available
  \usepackage[]{microtype}
  \UseMicrotypeSet[protrusion]{basicmath} % disable protrusion for tt fonts
}{}
\makeatletter
\@ifundefined{KOMAClassName}{% if non-KOMA class
  \IfFileExists{parskip.sty}{%
    \usepackage{parskip}
  }{% else
    \setlength{\parindent}{0pt}
    \setlength{\parskip}{6pt plus 2pt minus 1pt}}
}{% if KOMA class
  \KOMAoptions{parskip=half}}
\makeatother
\usepackage{xcolor}
\usepackage[left=1in,marginparwidth=2.0666666666667in,textwidth=4.1333333333333in,marginparsep=0.3in]{geometry}
\setlength{\emergencystretch}{3em} % prevent overfull lines
\setcounter{secnumdepth}{-\maxdimen} % remove section numbering
% Make \paragraph and \subparagraph free-standing
\ifx\paragraph\undefined\else
  \let\oldparagraph\paragraph
  \renewcommand{\paragraph}[1]{\oldparagraph{#1}\mbox{}}
\fi
\ifx\subparagraph\undefined\else
  \let\oldsubparagraph\subparagraph
  \renewcommand{\subparagraph}[1]{\oldsubparagraph{#1}\mbox{}}
\fi


\providecommand{\tightlist}{%
  \setlength{\itemsep}{0pt}\setlength{\parskip}{0pt}}\usepackage{longtable,booktabs,array}
\usepackage{calc} % for calculating minipage widths
% Correct order of tables after \paragraph or \subparagraph
\usepackage{etoolbox}
\makeatletter
\patchcmd\longtable{\par}{\if@noskipsec\mbox{}\fi\par}{}{}
\makeatother
% Allow footnotes in longtable head/foot
\IfFileExists{footnotehyper.sty}{\usepackage{footnotehyper}}{\usepackage{footnote}}
\makesavenoteenv{longtable}
\usepackage{graphicx}
\makeatletter
\def\maxwidth{\ifdim\Gin@nat@width>\linewidth\linewidth\else\Gin@nat@width\fi}
\def\maxheight{\ifdim\Gin@nat@height>\textheight\textheight\else\Gin@nat@height\fi}
\makeatother
% Scale images if necessary, so that they will not overflow the page
% margins by default, and it is still possible to overwrite the defaults
% using explicit options in \includegraphics[width, height, ...]{}
\setkeys{Gin}{width=\maxwidth,height=\maxheight,keepaspectratio}
% Set default figure placement to htbp
\makeatletter
\def\fps@figure{htbp}
\makeatother

\KOMAoption{captions}{tableheading}
\makeatletter
\makeatother
\makeatletter
\makeatother
\makeatletter
\@ifpackageloaded{caption}{}{\usepackage{caption}}
\AtBeginDocument{%
\ifdefined\contentsname
  \renewcommand*\contentsname{Зміст}
\else
  \newcommand\contentsname{Зміст}
\fi
\ifdefined\listfigurename
  \renewcommand*\listfigurename{Список Рисунків}
\else
  \newcommand\listfigurename{Список Рисунків}
\fi
\ifdefined\listtablename
  \renewcommand*\listtablename{Список Таблиць}
\else
  \newcommand\listtablename{Список Таблиць}
\fi
\ifdefined\figurename
  \renewcommand*\figurename{Рисунок}
\else
  \newcommand\figurename{Рисунок}
\fi
\ifdefined\tablename
  \renewcommand*\tablename{Таблиця}
\else
  \newcommand\tablename{Таблиця}
\fi
}
\@ifpackageloaded{float}{}{\usepackage{float}}
\floatstyle{ruled}
\@ifundefined{c@chapter}{\newfloat{codelisting}{h}{lop}}{\newfloat{codelisting}{h}{lop}[chapter]}
\floatname{codelisting}{Список}
\newcommand*\listoflistings{\listof{codelisting}{Список Каталогів}}
\makeatother
\makeatletter
\@ifpackageloaded{caption}{}{\usepackage{caption}}
\@ifpackageloaded{subcaption}{}{\usepackage{subcaption}}
\makeatother
\makeatletter
\@ifpackageloaded{tcolorbox}{}{\usepackage[many]{tcolorbox}}
\makeatother
\makeatletter
\@ifundefined{shadecolor}{\definecolor{shadecolor}{rgb}{.97, .97, .97}}
\makeatother
\makeatletter
\@ifpackageloaded{sidenotes}{}{\usepackage{sidenotes}}
\@ifpackageloaded{marginnote}{}{\usepackage{marginnote}}
\makeatother
\makeatletter
\makeatother
\ifLuaTeX
  \usepackage{selnolig}  % disable illegal ligatures
\fi
\IfFileExists{bookmark.sty}{\usepackage{bookmark}}{\usepackage{hyperref}}
\IfFileExists{xurl.sty}{\usepackage{xurl}}{} % add URL line breaks if available
\urlstyle{same} % disable monospaced font for URLs
\hypersetup{
  pdftitle={Мала академія наук України},
  colorlinks=true,
  linkcolor={blue},
  filecolor={Maroon},
  citecolor={Blue},
  urlcolor={Blue},
  pdfcreator={LaTeX via pandoc}}

\title{Мала академія наук України}
\author{}
\date{}

\begin{document}
\maketitle
\ifdefined\Shaded\renewenvironment{Shaded}{\begin{tcolorbox}[breakable, boxrule=0pt, sharp corners, borderline west={3pt}{0pt}{shadecolor}, frame hidden, enhanced, interior hidden]}{\end{tcolorbox}}\fi

\textbf{Мала́ Акаде́мія нау́к Украї́ни} --- освітня система, яка забезпечує
організацію і координацію науково-дослідної діяльності учнів, створює
умови для їх інтелектуального, духовного, творчого розвитку та
професійного самовизначення, сприяє нарощуванню наукового потенціалу
країни.

\begin{marginfigure}

{\centering \includegraphics{img/man.png}

}

\caption{Логотип Малої академії наук України}

\end{marginfigure}

\begin{longtable}[]{@{}
  >{\centering\arraybackslash}p{(\columnwidth - 2\tabcolsep) * \real{0.1944}}
  >{\raggedright\arraybackslash}p{(\columnwidth - 2\tabcolsep) * \real{0.8056}}@{}}
\toprule()
\begin{minipage}[b]{\linewidth}\centering
\textbf{Загальна інформація}
\end{minipage} & \begin{minipage}[b]{\linewidth}\raggedright
\end{minipage} \\
\midrule()
\endhead
\textbf{Країна} & {🇺🇦} Україна \\
\textbf{Штаб-квартира} & 04119, Київ, вул. Дегтярівська, 38-44 \\
\textbf{Відповідальний директор} &
\href{https://uk.wikipedia.org/wiki/\%D0\%9B\%D1\%96\%D1\%81\%D0\%BE\%D0\%B2\%D0\%B8\%D0\%B9_\%D0\%9E\%D0\%BA\%D1\%81\%D0\%B5\%D0\%BD_\%D0\%92\%D0\%B0\%D1\%81\%D0\%B8\%D0\%BB\%D1\%8C\%D0\%BE\%D0\%B2\%D0\%B8\%D1\%87}{Оксен
Лісовий}, кандидат філософських наук, лауреат Державної премії України в
галузі освіти \\
\href{http://man.gov.ua/}{man.gov.ua} & \\
\bottomrule()
\end{longtable}

\hypertarget{ux456ux441ux442ux43eux440ux456ux44f}{%
\section{Історія}\label{ux456ux441ux442ux43eux440ux456ux44f}}

\textbf{20--30-ті роки XX ст. --- початок формування гуртків}, на
заняття яких запрошувались вчені, що залучали дітей до
експериментальної, дослідницької, пошукової роботи у різних галузях
знань.

\textbf{1939 рік --- звернення Академії наук України} щодо
\textbf{підтримки роботи} з талановитими дітьми та учнівською молоддю
--- членами наукових гуртків.

\textbf{1947 рік --- початок роботи першого наукового товариства учнів
(НТУ)} --- «Товариства науки і техніки школярів» у м Києві.

\textbf{1950 рік} --- проведення в м. Києві \textbf{першої учнівської
науково-практичної конференції}.

\textbf{1960--90-ті роки} --- створення малих академій наук учнівської
молоді та наукових товариств учнів у різних регіонах України

\textbf{2000 рік} --- рішення Президії Національної академії наук
України «Про підвищення ролі Національної академії наук України в роботі
з творчою молоддю» і призначення С. Довгого президентом МАН.

\textbf{2004 рік} --- створений Позашкільний навчальний заклад Мала
академія наук учнівської молоді, на який були покладені функції
організаційно-методичного керівництва дослідницько-експериментальною
діяльністю учнів та завдання зі створення мережі регіональних
підрозділів.

\textbf{Лютий 2010 року} --- заклад реорганізовано в Український
державний центр «Мала академія наук України» Міністерства освіти і науки
України та Національної академії наук України.

\textbf{Квітень 2010 року} --- Державною програмою економічного і
соціального розвитку України на 2010 рік Український державний центр
«Мала академія наук України» визнано закладом, на базі якого створюється
загальнодержавна система пошуку і підтримки обдарованих дітей.

\textbf{Вересень 2010 року} --- відповідно до Указу Президента України,
Українському державному центру «Мала академія наук України» надано
статус національного і перейменовано у Національний центр «Мала академія
наук України».

\textbf{2015 рік} --- Закон України «Про наукову та науково-технічну
діяльність» визначає Малу академію наук України як мережу формування
інтелектуального капіталу нації та виховання майбутньої наукової зміни.
МАНУ має забезпечувати дослідницько-експериментальну, наукову,
конструкторську, винахідницьку та пошукову діяльність творчої молоді
України.

\textbf{2018 року} діяльність МАН одержала світове визнання: Мала
академія наук України отримала статус Центру наукової освіти II
категорії під егідою ЮНЕСКО. Відповідне рішення було одноголосно
прийняте на 39-й сесії Генеральної конференції ЮНЕСКО. МАН --- перша і
єдина в Україні освітня структура, що має такий престижний статус. Це
надає ексклюзивні можливості для сотень тисяч дітей і педагогів, сприяє
формуванню позитивного іміджу України на міжнародній арені та розвитку
освітньої дипломатії.

\textbf{У вересні 2018 року} Мала академія
\href{https://www.copernicus.eu/en/opportunities/education/copernicus-academy}{отримала
статус Академії Copernicus}. Ця мережа об'єднує 37 країн і спрямована на
популяризацію програми Європейського Союзу зі спостереження за Земною
поверхнею. Україна стала однією з перших країн --- не членів ЄС, яка
увійшла до цієї поважної організації. Статус Академії Copernicus
відкриває українським школярам і дослідникам доступ до даних 29-ти
європейських супутників, дає можливість брати участь у міжнародних
заходах у сфері дистанційного зондування Землі.

\hypertarget{ux447ux43bux435ux43dux441ux442ux432ux43e}{%
\section{Членство}\label{ux447ux43bux435ux43dux441ux442ux432ux43e}}

Члени МАН отримують статус слухачів, кандидатів і дійсних
членів\footnote{\href{https://uk.wikipedia.org/wiki/\%D0\%9C\%D0\%B0\%D0\%BB\%D0\%B0_\%D0\%B0\%D0\%BA\%D0\%B0\%D0\%B4\%D0\%B5\%D0\%BC\%D1\%96\%D1\%8F_\%D0\%BD\%D0\%B0\%D1\%83\%D0\%BA_\%D0\%A3\%D0\%BA\%D1\%80\%D0\%B0\%D1\%97\%D0\%BD\%D0\%B8\#CITEREF\%D0\%95\%D0\%A1\%D0\%A3_\%D0\%9C\%D0\%90\%D0\%9D}{ЕСУ\_МАН}}:

\begin{itemize}
\tightlist
\item
  \textbf{Слухачами} можуть стати всі охочі учні 6--11 кл.
  загальноосвітніх і професійно-технічних навчальних закладів, які
  виявляють цікавість до наукової діяльності, хочуть одержати додаткові
  знання в окремих галузях науки та беруть участь у роботі секції чи
  гуртка.
\item
  \textbf{Кандидатами} стають учні гуртків, секцій, які виявляють
  здібності до поглибленого вивчення наукових дисциплін поза шкільною
  програмою, схильні до проведення наукових досліджень, технічної
  творчості та виступають на конференціях, виставках, є призерами
  олімпіад.
\item
  \textbf{Дійсними членами} стають кандидати, які мають самостійні
  наукові праці та навчаються в наукових гуртках і секціях не менше 2-х
  років.
\end{itemize}

\hypertarget{ux43fux43eux441ux438ux43bux430ux43dux43dux44f}{%
\section{Посилання}\label{ux43fux43eux441ux438ux43bux430ux43dux43dux44f}}

\begin{itemize}
\tightlist
\item
  \href{http://man.gov.ua/}{Офіційний сайт}
\item
  \href{http://av.man.gov.ua/}{Сайт Асоціації випускників МАН}
\end{itemize}

\hypertarget{ux434ux438ux432.-ux442ux430ux43aux43eux436}{%
\section{Див. також}\label{ux434ux438ux432.-ux442ux430ux43aux43eux436}}

\begin{itemize}
\tightlist
\item
  \href{https://uk.wikipedia.org/wiki/\%D0\%9C\%D1\%83\%D0\%B7\%D0\%B5\%D0\%B9_\%D0\%BD\%D0\%B0\%D1\%83\%D0\%BA\%D0\%B8_(\%D0\%9A\%D0\%B8\%D1\%97\%D0\%B2)}{Музей
  науки Малої академії науки України}
\end{itemize}



\end{document}
